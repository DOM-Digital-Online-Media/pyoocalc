%
% API Documentation for API Documentation
% Module src.pylibratm
%
% Generated by epydoc 3.0.1
% [Wed Dec  9 19:50:11 2015]
%

%%%%%%%%%%%%%%%%%%%%%%%%%%%%%%%%%%%%%%%%%%%%%%%%%%%%%%%%%%%%%%%%%%%%%%%%%%%
%%                          Module Description                           %%
%%%%%%%%%%%%%%%%%%%%%%%%%%%%%%%%%%%%%%%%%%%%%%%%%%%%%%%%%%%%%%%%%%%%%%%%%%%

    \index{src \textit{(package)}!src.pylibratm \textit{(module)}|(}
\section{Module src.pylibratm}

    \label{src:pylibratm}
Python library for operating with LibreOffice/OpenOffice.org Calc.

Requirements for Ubuntu users:

sudo apt-get install python-uno

Optional requirements:

sudo apt-get install libreoffice-script-provider-python

\textbf{Author:} Yurii Puchkov



\textbf{Organization:} http://arilot.com/



\textbf{License:} GPL v3.0



\textbf{Contact:} panpuchkov@gmail.com




%%%%%%%%%%%%%%%%%%%%%%%%%%%%%%%%%%%%%%%%%%%%%%%%%%%%%%%%%%%%%%%%%%%%%%%%%%%
%%                           Class Description                           %%
%%%%%%%%%%%%%%%%%%%%%%%%%%%%%%%%%%%%%%%%%%%%%%%%%%%%%%%%%%%%%%%%%%%%%%%%%%%

    \index{src \textit{(package)}!src.pylibratm \textit{(module)}!src.pylibratm.Field \textit{(class)}|(}
\subsection{Class Field}

    \label{src:pylibratm:Field}
Document field.

Operations on the existing field.


%%%%%%%%%%%%%%%%%%%%%%%%%%%%%%%%%%%%%%%%%%%%%%%%%%%%%%%%%%%%%%%%%%%%%%%%%%%
%%                                Methods                                %%
%%%%%%%%%%%%%%%%%%%%%%%%%%%%%%%%%%%%%%%%%%%%%%%%%%%%%%%%%%%%%%%%%%%%%%%%%%%

  \subsubsection{Methods}

    \label{src:pylibratm:Field:__init__}
    \index{src \textit{(package)}!src.pylibratm \textit{(module)}!src.pylibratm.Field \textit{(class)}!src.pylibratm.Field.\_\_init\_\_ \textit{(method)}}

    \vspace{0.5ex}

\hspace{.8\funcindent}\begin{boxedminipage}{\funcwidth}

    \raggedright \textbf{\_\_init\_\_}(\textit{self}, \textit{fields}, \textit{name})

\setlength{\parskip}{2ex}
\setlength{\parskip}{1ex}
    \end{boxedminipage}

    \label{src:pylibratm:Field:fields}
    \index{src \textit{(package)}!src.pylibratm \textit{(module)}!src.pylibratm.Field \textit{(class)}!src.pylibratm.Field.fields \textit{(method)}}

    \vspace{0.5ex}

\hspace{.8\funcindent}\begin{boxedminipage}{\funcwidth}

    \raggedright \textbf{fields}(\textit{self})

    \vspace{-1.5ex}

    \rule{\textwidth}{0.5\fboxrule}
\setlength{\parskip}{2ex}
    Get fields object.

\setlength{\parskip}{1ex}
      \textbf{Return Value}
    \vspace{-1ex}

      \begin{quote}
      Fields object

      {\it (type=Fields)}

      \end{quote}

    \end{boxedminipage}

    \label{src:pylibratm:Field:is_null}
    \index{src \textit{(package)}!src.pylibratm \textit{(module)}!src.pylibratm.Field \textit{(class)}!src.pylibratm.Field.is\_null \textit{(method)}}

    \vspace{0.5ex}

\hspace{.8\funcindent}\begin{boxedminipage}{\funcwidth}

    \raggedright \textbf{is\_null}(\textit{self})

    \vspace{-1.5ex}

    \rule{\textwidth}{0.5\fboxrule}
\setlength{\parskip}{2ex}
    Checking if a field is null.

\setlength{\parskip}{1ex}
      \textbf{Return Value}
    \vspace{-1ex}

      \begin{quote}
      Value insertion result

      {\it (type=boolean)}

      \end{quote}

    \end{boxedminipage}

    \label{src:pylibratm:Field:set_value}
    \index{src \textit{(package)}!src.pylibratm \textit{(module)}!src.pylibratm.Field \textit{(class)}!src.pylibratm.Field.set\_value \textit{(method)}}

    \vspace{0.5ex}

\hspace{.8\funcindent}\begin{boxedminipage}{\funcwidth}

    \raggedright \textbf{set\_value}(\textit{self}, \textit{value}, \textit{column}={\tt 0}, \textit{row}={\tt 0})

    \vspace{-1.5ex}

    \rule{\textwidth}{0.5\fboxrule}
\setlength{\parskip}{2ex}
    Set filed value at position Column/Row

\setlength{\parskip}{1ex}
      \textbf{Parameters}
      \vspace{-1ex}

      \begin{quote}
        \begin{Ventry}{xxxxxx}

          \item[value]

          Cell value

            {\it (type=string)}

          \item[column]

          column index

            {\it (type=integer)}

          \item[row]

          row index

            {\it (type=integer)}

        \end{Ventry}

      \end{quote}

      \textbf{Return Value}
    \vspace{-1ex}

      \begin{quote}
      Value insertion result

      {\it (type=boolean)}

      \end{quote}

    \end{boxedminipage}

    \label{src:pylibratm:Field:value}
    \index{src \textit{(package)}!src.pylibratm \textit{(module)}!src.pylibratm.Field \textit{(class)}!src.pylibratm.Field.value \textit{(method)}}

    \vspace{0.5ex}

\hspace{.8\funcindent}\begin{boxedminipage}{\funcwidth}

    \raggedright \textbf{value}(\textit{self}, \textit{column}={\tt 0}, \textit{row}={\tt 0})

    \vspace{-1.5ex}

    \rule{\textwidth}{0.5\fboxrule}
\setlength{\parskip}{2ex}
    Get filed value at position Column/Row

\setlength{\parskip}{1ex}
      \textbf{Parameters}
      \vspace{-1ex}

      \begin{quote}
        \begin{Ventry}{xxxxxx}

          \item[column]

          column index

            {\it (type=integer)}

          \item[row]

          row index

            {\it (type=integer)}

        \end{Ventry}

      \end{quote}

      \textbf{Return Value}
    \vspace{-1ex}

      \begin{quote}
      Document cell value in string format. Regardless of document
      cell type.

      {\it (type=string)}

      \end{quote}

    \end{boxedminipage}

    \label{src:pylibratm:Field:insert_row}
    \index{src \textit{(package)}!src.pylibratm \textit{(module)}!src.pylibratm.Field \textit{(class)}!src.pylibratm.Field.insert\_row \textit{(method)}}

    \vspace{0.5ex}

\hspace{.8\funcindent}\begin{boxedminipage}{\funcwidth}

    \raggedright \textbf{insert\_row}(\textit{self}, \textit{row}={\tt 1}, \textit{step}={\tt 1}, \textit{num\_columns}={\tt 1}, \textit{offset}={\tt 0})

    \vspace{-1.5ex}

    \rule{\textwidth}{0.5\fboxrule}
\setlength{\parskip}{2ex}
    Insert rows

    Insert new rows at the specified position relatively to cell. After the
    new row insertion the content of the current rows is copied to the new 
    rows.

\setlength{\parskip}{1ex}
      \textbf{Parameters}
      \vspace{-1ex}

      \begin{quote}
        \begin{Ventry}{xxxxxxxxxxx}

          \item[row]

          Row index. Default value=1.

            {\it (type=integer)}

          \item[step]

          Default value = 1.    // FIXME

            {\it (type=integer)}

          \item[num\_columns]

          Number of columns to copy. Default value=1

            {\it (type=integer)}

          \item[offset]

          Rows offset. Relatively to current field. Default value=0

            {\it (type=integer)}

        \end{Ventry}

      \end{quote}

      \textbf{Return Value}
    \vspace{-1ex}

      \begin{quote}
      Operation result

      {\it (type=boolean)}

      \end{quote}

    \end{boxedminipage}

    \label{src:pylibratm:Field:insert_column}
    \index{src \textit{(package)}!src.pylibratm \textit{(module)}!src.pylibratm.Field \textit{(class)}!src.pylibratm.Field.insert\_column \textit{(method)}}

    \vspace{0.5ex}

\hspace{.8\funcindent}\begin{boxedminipage}{\funcwidth}

    \raggedright \textbf{insert\_column}(\textit{self}, \textit{column}, \textit{step}={\tt 1}, \textit{num\_rows}={\tt 1}, \textit{offset}={\tt 0})

    \vspace{-1.5ex}

    \rule{\textwidth}{0.5\fboxrule}
\setlength{\parskip}{2ex}
    Insert rows

    Insert new column at the specified position relatively to cell. After 
    the new column insertion the content of the current columns is copied 
    to the new columns.

\setlength{\parskip}{1ex}
      \textbf{Parameters}
      \vspace{-1ex}

      \begin{quote}
        \begin{Ventry}{xxxxxxxx}

          \item[column]

          Row index. Default value=1.

            {\it (type=integer)}

          \item[step]

          Default value = 1.    // FIXME

            {\it (type=integer)}

          \item[num\_rows]

          Number of columns to copy. Default value=1

            {\it (type=integer)}

          \item[offset]

          Rows offset. Relatively to current field. Default value=0

            {\it (type=integer)}

        \end{Ventry}

      \end{quote}

      \textbf{Return Value}
    \vspace{-1ex}

      \begin{quote}
      Operation result

      {\it (type=boolean)}

      \end{quote}

    \end{boxedminipage}

    \label{src:pylibratm:Field:remove}
    \index{src \textit{(package)}!src.pylibratm \textit{(module)}!src.pylibratm.Field \textit{(class)}!src.pylibratm.Field.remove \textit{(method)}}

    \vspace{0.5ex}

\hspace{.8\funcindent}\begin{boxedminipage}{\funcwidth}

    \raggedright \textbf{remove}(\textit{self})

    \vspace{-1.5ex}

    \rule{\textwidth}{0.5\fboxrule}
\setlength{\parskip}{2ex}
    Remove field name.

    Remove field name from the document. Not implemented yet.

\setlength{\parskip}{1ex}
    \end{boxedminipage}

    \index{src \textit{(package)}!src.pylibratm \textit{(module)}!src.pylibratm.Field \textit{(class)}|)}

%%%%%%%%%%%%%%%%%%%%%%%%%%%%%%%%%%%%%%%%%%%%%%%%%%%%%%%%%%%%%%%%%%%%%%%%%%%
%%                           Class Description                           %%
%%%%%%%%%%%%%%%%%%%%%%%%%%%%%%%%%%%%%%%%%%%%%%%%%%%%%%%%%%%%%%%%%%%%%%%%%%%

    \index{src \textit{(package)}!src.pylibratm \textit{(module)}!src.pylibratm.Fields \textit{(class)}|(}
\subsection{Class Fields}

    \label{src:pylibratm:Fields}
Document fields. Search and manage fields.


%%%%%%%%%%%%%%%%%%%%%%%%%%%%%%%%%%%%%%%%%%%%%%%%%%%%%%%%%%%%%%%%%%%%%%%%%%%
%%                                Methods                                %%
%%%%%%%%%%%%%%%%%%%%%%%%%%%%%%%%%%%%%%%%%%%%%%%%%%%%%%%%%%%%%%%%%%%%%%%%%%%

  \subsubsection{Methods}

    \label{src:pylibratm:Fields:__init__}
    \index{src \textit{(package)}!src.pylibratm \textit{(module)}!src.pylibratm.Fields \textit{(class)}!src.pylibratm.Fields.\_\_init\_\_ \textit{(method)}}

    \vspace{0.5ex}

\hspace{.8\funcindent}\begin{boxedminipage}{\funcwidth}

    \raggedright \textbf{\_\_init\_\_}(\textit{self}, \textit{template})

\setlength{\parskip}{2ex}
\setlength{\parskip}{1ex}
    \end{boxedminipage}

    \label{src:pylibratm:Fields:template}
    \index{src \textit{(package)}!src.pylibratm \textit{(module)}!src.pylibratm.Fields \textit{(class)}!src.pylibratm.Fields.template \textit{(method)}}

    \vspace{0.5ex}

\hspace{.8\funcindent}\begin{boxedminipage}{\funcwidth}

    \raggedright \textbf{template}(\textit{self})

    \vspace{-1.5ex}

    \rule{\textwidth}{0.5\fboxrule}
\setlength{\parskip}{2ex}
    Get template object.

\setlength{\parskip}{1ex}
      \textbf{Return Value}
    \vspace{-1ex}

      \begin{quote}
      Template object

      {\it (type=Template)}

      \end{quote}

    \end{boxedminipage}

    \label{src:pylibratm:Fields:field}
    \index{src \textit{(package)}!src.pylibratm \textit{(module)}!src.pylibratm.Fields \textit{(class)}!src.pylibratm.Fields.field \textit{(method)}}

    \vspace{0.5ex}

\hspace{.8\funcindent}\begin{boxedminipage}{\funcwidth}

    \raggedright \textbf{field}(\textit{self}, \textit{name})

    \vspace{-1.5ex}

    \rule{\textwidth}{0.5\fboxrule}
\setlength{\parskip}{2ex}
    Get document field by name

\setlength{\parskip}{1ex}
      \textbf{Parameters}
      \vspace{-1ex}

      \begin{quote}
        \begin{Ventry}{xxxx}

          \item[name]

          Field name

            {\it (type=string)}

        \end{Ventry}

      \end{quote}

      \textbf{Return Value}
    \vspace{-1ex}

      \begin{quote}
      Field object

      {\it (type=Field object)}

      \end{quote}

    \end{boxedminipage}

    \label{src:pylibratm:Fields:insert_spreadsheet}
    \index{src \textit{(package)}!src.pylibratm \textit{(module)}!src.pylibratm.Fields \textit{(class)}!src.pylibratm.Fields.insert\_spreadsheet \textit{(method)}}

    \vspace{0.5ex}

\hspace{.8\funcindent}\begin{boxedminipage}{\funcwidth}

    \raggedright \textbf{insert\_spreadsheet}(\textit{self}, \textit{name}, \textit{index})

    \vspace{-1.5ex}

    \rule{\textwidth}{0.5\fboxrule}
\setlength{\parskip}{2ex}
    Insert spreadsheet.

    Not implemented yet.

\setlength{\parskip}{1ex}
    \end{boxedminipage}

    \label{src:pylibratm:Fields:add}
    \index{src \textit{(package)}!src.pylibratm \textit{(module)}!src.pylibratm.Fields \textit{(class)}!src.pylibratm.Fields.add \textit{(method)}}

    \vspace{0.5ex}

\hspace{.8\funcindent}\begin{boxedminipage}{\funcwidth}

    \raggedright \textbf{add}(\textit{self}, \textit{name})

    \vspace{-1.5ex}

    \rule{\textwidth}{0.5\fboxrule}
\setlength{\parskip}{2ex}
    Get field.

    Not implemented yet.

\setlength{\parskip}{1ex}
    \end{boxedminipage}

    \label{src:pylibratm:Fields:count}
    \index{src \textit{(package)}!src.pylibratm \textit{(module)}!src.pylibratm.Fields \textit{(class)}!src.pylibratm.Fields.count \textit{(method)}}

    \vspace{0.5ex}

\hspace{.8\funcindent}\begin{boxedminipage}{\funcwidth}

    \raggedright \textbf{count}(\textit{self})

    \vspace{-1.5ex}

    \rule{\textwidth}{0.5\fboxrule}
\setlength{\parskip}{2ex}
    Get fields count.

    Not implemented yet.

\setlength{\parskip}{1ex}
    \end{boxedminipage}

    \index{src \textit{(package)}!src.pylibratm \textit{(module)}!src.pylibratm.Fields \textit{(class)}|)}

%%%%%%%%%%%%%%%%%%%%%%%%%%%%%%%%%%%%%%%%%%%%%%%%%%%%%%%%%%%%%%%%%%%%%%%%%%%
%%                           Class Description                           %%
%%%%%%%%%%%%%%%%%%%%%%%%%%%%%%%%%%%%%%%%%%%%%%%%%%%%%%%%%%%%%%%%%%%%%%%%%%%

    \index{src \textit{(package)}!src.pylibratm \textit{(module)}!src.pylibratm.Template \textit{(class)}|(}
\subsection{Class Template}

    \label{src:pylibratm:Template}

%%%%%%%%%%%%%%%%%%%%%%%%%%%%%%%%%%%%%%%%%%%%%%%%%%%%%%%%%%%%%%%%%%%%%%%%%%%
%%                                Methods                                %%
%%%%%%%%%%%%%%%%%%%%%%%%%%%%%%%%%%%%%%%%%%%%%%%%%%%%%%%%%%%%%%%%%%%%%%%%%%%

  \subsubsection{Methods}

    \label{src:pylibratm:Template:__init__}
    \index{src \textit{(package)}!src.pylibratm \textit{(module)}!src.pylibratm.Template \textit{(class)}!src.pylibratm.Template.\_\_init\_\_ \textit{(method)}}

    \vspace{0.5ex}

\hspace{.8\funcindent}\begin{boxedminipage}{\funcwidth}

    \raggedright \textbf{\_\_init\_\_}(\textit{self}, \textit{connection\_string}={\tt \texttt{...}})

\setlength{\parskip}{2ex}
\setlength{\parskip}{1ex}
    \end{boxedminipage}

    \label{src:pylibratm:Template:document}
    \index{src \textit{(package)}!src.pylibratm \textit{(module)}!src.pylibratm.Template \textit{(class)}!src.pylibratm.Template.document \textit{(method)}}

    \vspace{0.5ex}

\hspace{.8\funcindent}\begin{boxedminipage}{\funcwidth}

    \raggedright \textbf{document}(\textit{self})

    \vspace{-1.5ex}

    \rule{\textwidth}{0.5\fboxrule}
\setlength{\parskip}{2ex}
    LibreOffice/OpenOffice Calc document object.

    Required for Fileds and Field classes. Do not use it directly.

\setlength{\parskip}{1ex}
    \end{boxedminipage}

    \label{src:pylibratm:Template:fields}
    \index{src \textit{(package)}!src.pylibratm \textit{(module)}!src.pylibratm.Template \textit{(class)}!src.pylibratm.Template.fields \textit{(method)}}

    \vspace{0.5ex}

\hspace{.8\funcindent}\begin{boxedminipage}{\funcwidth}

    \raggedright \textbf{fields}(\textit{self})

    \vspace{-1.5ex}

    \rule{\textwidth}{0.5\fboxrule}
\setlength{\parskip}{2ex}
    Get fields document's object.

\setlength{\parskip}{1ex}
      \textbf{Return Value}
    \vspace{-1ex}

      \begin{quote}
      Fields object

      {\it (type=Fields)}

      \end{quote}

    \end{boxedminipage}

    \label{src:pylibratm:Template:version}
    \index{src \textit{(package)}!src.pylibratm \textit{(module)}!src.pylibratm.Template \textit{(class)}!src.pylibratm.Template.version \textit{(method)}}

    \vspace{0.5ex}

\hspace{.8\funcindent}\begin{boxedminipage}{\funcwidth}

    \raggedright \textbf{version}(\textit{self})

    \vspace{-1.5ex}

    \rule{\textwidth}{0.5\fboxrule}
\setlength{\parskip}{2ex}
    Get library version.

\setlength{\parskip}{1ex}
      \textbf{Return Value}
    \vspace{-1ex}

      \begin{quote}
      PyLibra version

      {\it (type=string)}

      \end{quote}

    \end{boxedminipage}

    \label{src:pylibratm:Template:close_document}
    \index{src \textit{(package)}!src.pylibratm \textit{(module)}!src.pylibratm.Template \textit{(class)}!src.pylibratm.Template.close\_document \textit{(method)}}

    \vspace{0.5ex}

\hspace{.8\funcindent}\begin{boxedminipage}{\funcwidth}

    \raggedright \textbf{close\_document}(\textit{self})

    \vspace{-1.5ex}

    \rule{\textwidth}{0.5\fboxrule}
\setlength{\parskip}{2ex}
    Close document.

    Close current document. Not implemented yet.

\setlength{\parskip}{1ex}
    \end{boxedminipage}

    \label{src:pylibratm:Template:save_document}
    \index{src \textit{(package)}!src.pylibratm \textit{(module)}!src.pylibratm.Template \textit{(class)}!src.pylibratm.Template.save\_document \textit{(method)}}

    \vspace{0.5ex}

\hspace{.8\funcindent}\begin{boxedminipage}{\funcwidth}

    \raggedright \textbf{save\_document}(\textit{self}, \textit{doc\_name}={\tt ""})

    \vspace{-1.5ex}

    \rule{\textwidth}{0.5\fboxrule}
\setlength{\parskip}{2ex}
    Save document.

\setlength{\parskip}{1ex}
      \textbf{Parameters}
      \vspace{-1ex}

      \begin{quote}
        \begin{Ventry}{xxxxxxxx}

          \item[doc\_name]

          Document name. If no document name defined the current name is 
          used.

            {\it (type=string)}

        \end{Ventry}

      \end{quote}

      \textbf{Return Value}
    \vspace{-1ex}

      \begin{quote}
      Operation result

      {\it (type=boolean)}

      \end{quote}

    \end{boxedminipage}

    \label{src:pylibratm:Template:open_document}
    \index{src \textit{(package)}!src.pylibratm \textit{(module)}!src.pylibratm.Template \textit{(class)}!src.pylibratm.Template.open\_document \textit{(method)}}

    \vspace{0.5ex}

\hspace{.8\funcindent}\begin{boxedminipage}{\funcwidth}

    \raggedright \textbf{open\_document}(\textit{self}, \textit{doc\_name})

    \vspace{-1.5ex}

    \rule{\textwidth}{0.5\fboxrule}
\setlength{\parskip}{2ex}
    Open document.

\setlength{\parskip}{1ex}
      \textbf{Parameters}
      \vspace{-1ex}

      \begin{quote}
        \begin{Ventry}{xxxxxxxx}

          \item[doc\_name]

          Document name.

            {\it (type=string)}

        \end{Ventry}

      \end{quote}

      \textbf{Return Value}
    \vspace{-1ex}

      \begin{quote}
      Operation result

      {\it (type=boolean)}

      \end{quote}

    \end{boxedminipage}

    \label{src:pylibratm:Template:new_document}
    \index{src \textit{(package)}!src.pylibratm \textit{(module)}!src.pylibratm.Template \textit{(class)}!src.pylibratm.Template.new\_document \textit{(method)}}

    \vspace{0.5ex}

\hspace{.8\funcindent}\begin{boxedminipage}{\funcwidth}

    \raggedright \textbf{new\_document}(\textit{self}, \textit{doc\_name})

    \vspace{-1.5ex}

    \rule{\textwidth}{0.5\fboxrule}
\setlength{\parskip}{2ex}
    Create new document.

    Create new document. Not implemented yet.

\setlength{\parskip}{1ex}
    \end{boxedminipage}

    \index{src \textit{(package)}!src.pylibratm \textit{(module)}!src.pylibratm.Template \textit{(class)}|)}
    \index{src \textit{(package)}!src.pylibratm \textit{(module)}|)}
